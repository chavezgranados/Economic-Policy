\documentclass[letterpaper, 11pt]{extarticle}
% \usepackage{fontspec}

% Document parameters
\usepackage[spanish]{babel}
\decimalpoint %separador punto
\usepackage[margin = 1in]{geometry}

% Packages for math
\usepackage{mathrsfs}
\usepackage{amsfonts}
\usepackage{amsmath}
\usepackage{amsthm}
\usepackage{amssymb}
\usepackage{physics}
\usepackage{dsfont}
\usepackage{esint}

% Packages for writing
\usepackage{enumerate}
\usepackage[shortlabels]{enumitem}
\usepackage{framed}
\usepackage{csquotes}

% Miscellaneous packages
\usepackage{float}
\usepackage{tabularx}
\usepackage{xcolor}
\usepackage{multicol}
\usepackage{subcaption}
\usepackage{caption}
\captionsetup{format = hang, margin = 10pt, font = small, labelfont = bf}



%///////////////////////////  FORMATO  //////////////////////////

\usepackage{titlesec}
\usepackage[many]{tcolorbox}

% Adjust spacing after the chapter title
\titlespacing*{\chapter}{0cm}{-2.0cm}{0.50cm}
\titlespacing*{\section}{0cm}{0.50cm}{0.25cm}

% Indent 
\setlength{\parindent}{0pt}
\setlength{\parskip}{1ex}

% \numberwithin{equation}{section}

\newtcbtheorem[]{problem}{Problema}%
    {enhanced,
    colback = lime!8, %white,
    colbacktitle = lime!35,
    coltitle = black,
    boxrule = 0pt,
    frame hidden,
    borderline west = {0.5mm}{0.0mm}{black},
    fonttitle = \bfseries\sffamily,
    breakable,
    before skip = 3ex,
    after skip = 3ex
}{problem}

\tcbuselibrary{skins, breakable}
%\pagecolor{yellow!7}



%///////////////////////////  CONTENIDO  //////////////////////////
\begin{document}


\begin{Large}
    \textsf{\textbf{PS01 -- Economic Policy}}
    
    \normalsize \textsf{Lecturer:} \text{Luis Chávez} 
    \vspace{0.2cm} \hrule
\end{Large}
\vspace{0.2cm}
Los siguientes ejercicios permiten medir la capacidad analítica y procedimental. Se sugiere resolverlos en forma ascendente.




%----------------------------------------------------------------
\begin{problem}{DGE determinístico}{}

\textbf{a) El modelo}:

La familia representativa resuelve
\begin{equation}
    \underset{\{C_t,I_t,O_t\}}{\max}=\sum_{t=0}^{\infty} \beta^t[\theta\ln(C_t)+(1-\theta)\ln(1-L_t)]; \ \ \ 0<\beta<1
\end{equation}

sujeto a
\begin{equation}
    C_t+I_t=w_tL_t+r_tK_t
\end{equation}
\begin{equation}
    K_{t+1}=I_t+(1-\delta)K_t, \ \ \ \delta>0
\end{equation}

La firma representativa resuelve:
\begin{equation}
    \underset{\{K_t,L_t\}}{\max} \pi_t=Y_t-w_tL_t-r_tK_t
\end{equation}
s.a
\begin{equation}
    Y_t=A_tK_t^{1-\alpha}L_t^{\alpha}
\end{equation}

\textbf{b) Solución}

El langriano de la familia será:
\begin{equation}
    \mathcal{L}=\sum_{t=0}^{\infty} \left\{\beta^t[\theta\ln(C_t)+(1-\theta)\ln(1-L_t)]-\lambda_t[C_t+K_{t+1}-(1-\delta)K_t-w_tL_t-r_tK_t] \right \}
\end{equation}

FOC:

\begin{align}
    \underline{C_t|} & \hspace{.5cm} \beta^t\left\{\theta \dfrac{1}{C_t}-\lambda_t\right\}=0 \\
    \underline{L_t|}  & \hspace{.5cm} \beta^t\left\{-(1-\theta)\dfrac{1}{1-L_t}+w_t\lambda_t  \right\}=0 \\
    \underline{K_t|} & \hspace{.5cm} \beta^t\left\{\lambda_t(1-\delta+r_t)\right\}-\beta^{t-1}\lambda_{t-1}=0 \\
    \underline{\lambda_t|} & \hspace{.5cm} \beta^t\left\{C_t+K_{t+1}-(1-\delta)K_t-w_tL_t-r_tK_t\right\}=0
\end{align}

De (7) en (8) se tiene
\begin{equation*}
    (1-\theta)\dfrac{1}{1-L_t}=w_t\dfrac{\theta}{C_t}
\end{equation*}

Organizando, se tiene la ecuación de euler de la relación marginal de sustitución entre consumo y ocio:
\begin{equation}
    w_t=\dfrac{1-\theta}{\theta}\dfrac{C_t}{1-L_t}
\end{equation}

De (7) en (9), se tiene
\begin{equation*}
    \beta \dfrac{\theta}{C_t}(1-\delta+r_t)=\dfrac{\theta}{C_{t-1}}
\end{equation*}

Organizando, se tiene la ecuación de Euler del ratio del consumo:
\begin{equation}
    \dfrac{C_t}{C_{t-1}}=\beta[1-\delta+r_t]
\end{equation}

De (10) se tiene,
\begin{equation}
    C_t+K_{t+1}-(1-\delta)K_t=w_tL_t+r_tK_t
\end{equation}

Por su parte, la firma resuelve una expresión irrestricta si de (4) y (5) se escribe:
\begin{equation}
    \underset{\{K_t,L_t\}}{\max} \pi_t=A_tK_t^{1-\alpha}L_t^{\alpha}-w_tL_t-r_tK_t
\end{equation}

FOC:
\begin{align}
    \underline{K_t|} & \hspace{.5cm} (1-\alpha)A_tK^{-\alpha}L_t^{\alpha}-r_t=0 \\
    \underline{L_t|} & \hspace{.5cm} \alpha A_tK^{1-\alpha}L_t^{\alpha-1}-w_t=0
\end{align}

De (15), se tiene la ecuación de demanda de capital:
\begin{equation}
    r_t=(1-\alpha)A_tK^{-\alpha}L_t^{\alpha}=(1-\alpha)\dfrac{Y_t}{K_t}
\end{equation}

De (16), se tiene la ecuación de demanda laboral:
\begin{equation}
    w_t=\alpha A_tK^{1-\alpha}L_t^{\alpha-1}=\alpha\dfrac{Y_t}{L_t}
\end{equation}

Ahora, (17) y (18) en (13), se tiene:
$$C_t+K_{t+1}-(1-\delta)K_t=\left[\alpha A_tK^{1-\alpha}L_t^{\alpha-1}\right]L_t+\left[(1-\alpha)A_tK^{-\alpha}L_t^{\alpha}\right]K_t$$

De donde se tiene la ecuación en diferencia (limpieza):
\begin{equation}
    C_t+K_{t+1}-(1-\delta)K_t=A_tK^{1-\alpha}L_t^{\alpha}
\end{equation}

Asimismo, se puede reemplazar (18) en (11) y se tiene
\begin{equation}
    \alpha A_tK^{1-\alpha}L_t^{\alpha-1}=\dfrac{1-\theta}{\theta}\dfrac{C_t}{1-L_t}
\end{equation}

También se puede reemplazar (18) en (12) y se tiene ora ecuación en diferencias:
\begin{equation}
    \dfrac{C_t}{C_{t-1}}=\beta[1-\delta+(1-\alpha)A_tK^{-\alpha}L_t^{\alpha}]
\end{equation}


Por lo tanto, la solución competitiva estará dada por las ecuaciones (19), (20) y (21).

\begin{equation}
\left\{
\begin{array}{l}
    C_t + K_{t+1} - (1-\delta)K_t = A_t K^{1-\alpha} L_t^{\alpha} \\
    \alpha A_t K^{1-\alpha} L_t^{\alpha-1} = \frac{1-\theta}{\theta} \frac{C_t}{1-L_t} \\
    \frac{C_t}{C_{t-1}} = \beta \left[1 - \delta + (1 - \alpha) A_t K^{-\alpha} L_t^{\alpha} \right]
\end{array}
\right.
\end{equation}


\end{problem}


\begin{problem}{DGE centralizado}{}
Resuelva el modelo centralizado del problema 1.

\end{problem}


%----------------------------------------------------------------
\begin{problem}{DGE con dinero}{}
Sea un MIU sin incertidumbre de horizonte finito, donde el hogar representativo sólo vive 3 períodos $\{t\}_0^2$. El hogar resuelve su problema eligiendo consumo, saldos reales de dinero y bonos reales. En concreto,
\begin{equation}
    \max_{\{c_t, m_t, b_t\}_{t=0}^{2}} \sum_{t=0}^{2} \beta^t \left[ \ln(c_t) + \gamma \ln(m_t) \right], \quad \gamma > 0, \ \beta \in (0,1)
\end{equation}
s.a
\begin{equation}
    c_t + m_t + b_t = w_t + \tau_t + b_{t-1} \frac{1 + i_{t-1}}{1 + \pi_t} + m_{t-1} \frac{1}{1 + \pi_t}, \quad \{t\}_0^2
\end{equation}

$$m_t \geq 0, \quad \{t\}_0^2$$

$$m_{-1}, b_{-1} \ \text{dado}$$

El hogar trabaja para su propia firma y percibe salarios ($w_t$). Otra fuente de ingreso son las transferencias no condicionadas ($\tau$). Los otros precios, $r_t$ y $\pi_t$, también están dados.

\begin{enumerate}[a)]
    \item Explique la necesidad de imponer la restricción de no negatividad $b_2>0$.
    \item Escriba el lagrangiano del hogar.
    \item Resuelva las FOC.
    \item Halle las ecuaciones de Euler e interprete.
\end{enumerate}
\end{problem}




\end{document}


%----------------------------------------------------------------
\begin{problem}{Review estadístico}{}
Sean..........

\end{problem}